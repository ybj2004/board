\chapter{采购管理规范}\label{app:appendix2-1}
\section{四川三联卷烟材料有限公司采购管理规定}
\begin{center}
\textbf{第一章 \quad  总  则}
\end{center}

\textbf{第一条}\quad 为规范企业采购行为,加强对物资、工程和服务采购的管理与监督,依据《川渝中烟工业有限责任公司采购管理规范》及国家相关法律、法规、规章,制订本规定。

\textbf{第二条}\quad 本规定适用于四川三联卷烟材料有限公司(以下简称三联公司)需要进行立项的采购类经济事项活动管理。需要立项的采购项目,以三联公司“工程投资、物资采购、宣传促销”管理委员会(以下简称公司“三项工作”管委会)审定的《四川三联卷烟材料有限公司采购目录》为准。

\textbf{第三条}\quad 本规定所称采购,是指三联公司以合同方式有偿取得物资、工程(含信息化项目)和服务的行为,包括购买、租赁、委托、雇用等;

所称物资,是指各种形态和种类的物品,分为烟用物资和非烟用物资。烟用物资主要包括滤棒原辅材料、烟草物流配送物资、烟机及烟机零配件等;非烟用物资主要包括原材料、燃料、运输工具、办公、劳保、生活、五金、设备等其它产品。

所称工程,是指建设工程,包括建筑物和构筑物的新建、改建、扩建及其装修、拆除、修缮,以及与建设工程相关的勘察、设计、施工、监理等。信息化项目还包括软件、硬件及信息系统采购。

所称服务,是指除物资和工程以外的其他采购对象,包括宣传促销服务、管理咨询服务、科研开发服务、信息网络开发服务、金融保险服务、运输服务、维修与维护服务及其他各类专业服务等。

\textbf{第四条}\quad 采购工作遵循公开透明原则、公平竞争原则、公正原则和诚实信用原则、痕迹管理原则、监管到位原则和公开招标为主要采购方式的原则等。

\textbf{第五条}\quad 采购工作必须遵守国家相关法律和法规并依法接受监督。

\begin{center}
\textbf{第二章 \quad  采购管理机构与职责}
\end{center}

\textbf{第六条}\quad 三联公司董事会是本公司采购管理的决策机构,负责审定本公司年度采购计划及其调整方案、审定项目采购方式。

\textbf{第七条}\quad 公司“三项工作”管理委员会是董事会的咨询和执行机构。“三项工作”管委会的人员由公司领导班子全体成员、相关部门主要负责人以及职工代表组成。

\textbf{第八条}\quad  “三项工作”管委会的主要职责:审定本单位采购工作管理制度和程序;审定公司《集中采购目录》;审议年度采购计划;审议项目采购方式;审定项目立项申请、招标文件、谈判文件等;审定中选供应商;确定候选招标代理机构名单;指导和监督采管办的工作。

年度采购计划、采购方式经“三项工作”管委会审议后,需报董事会审议批准。

\textbf{第九条 }\quad “三项工作”管委会下设公司采购管理办公室(以下简称采管办),采管办挂靠办公室合署办公,采管办主任由公司办主要负责人兼任,成员由有关部门人员组成。

采管办职责:严格执行国家和行业关于采购管理方面的法律、法规,贯彻和落实公司董事会、管理委员会的各项审批和决议;负责起草采购工作管理制度和程序;编制公司《集中采购目录》;审核、汇总年度采购需求,编制年度采购计划;抽选招标代理机构;牵头组织审核立项申请、招标文件、谈判文件、询价文件等;审核中选供应商;组织自行招标采购;发布中选供应商。

\textbf{第十条 }\quad 集中采购部门职责:本单位集中采购部门为供应科,负责公司物资采购;负责拟定年度物资采购计划及集中采购方案,报采管办汇总;组织编制集中采购招标文件、谈判文件、询价文件;组织集中采购商务谈判工作;组织实施供应商的资质认证和考核评价,负责管理供应商库;向采管办提交中选供应商名单;与中选供应商签订集中采购协议;负责公司采购合同管理;负责编制各类物资年、季、月的采购统计报表,保存集中采购文档、资料和记录。

\textbf{第十一条}\quad  监督管理部门职责:
公司整顿办为监督工作牵头部门,公司整顿办设在安全保卫科,负责组织采购监督工作小组,协调及负责监督工作小组工作。监督工作小组由整顿办、财务审计科、政工科(纪检)相关人员组成,对采购项目进行同步监督。

具体工作职责为:监督采购项目计划是否按照规定经过公司决策机构审批同意;监督采购项目的采购方式、组织形式是否按照规定经过公司决策机构审批同意;监督采购方式、组织形式是否与批准确定的采购方式、组织形式相符;监督采购项目是否符合国家或行业规定的程序实施采购活动,并根据需要进行调查核实;参与自行组织招标、竞争性谈判、询价项目评审专家的抽取;对参加自行组织招标、竞争性谈判、询价项目的供应商授权代表进行身份核查;负责对采购活动实施程序性、全过程跟踪或实质性审计;对公司批准的采购项目的招标文件、采购谈判文件、询价方案等相关文件进行书面审核,提出审核意见;对采购活动中有关法律法规问题在我方律师配合下进行咨询和指导;其他需要监督的事项。

职工代表:职工代表行使公司职代会赋予的知情权、参与权、监督权,全程参与自行组织招标开、评标的现场监督工作,并对招标项目开、评标现场是否公开、透明,同步监督是否到位进行监督。具体项目参与监督的职工代表,由公司工会确定,并负责通知其具体参与时间。

\begin{center}
\textbf{第三章 \quad  集中采购目录和采购计划编制}
\end{center}

\textbf{第十二条}\quad  本单位采购组织形式为集中采购。
集中采购,是指通用物资、工程和服务,由供应科依据各部门采购需求,统一组织的采购。
集中采购目录编制:各部门在每年11月15日前依据下一年度采购需求编制本部门的采购目录,提交采管办初审。采管办按照采购组织形式和采购需求编制公司《采购目录》,并提交公司“三项工作”管理委员会审定。采购目录需包括采购部门、采购类别、采购项目名称等。

\textbf{第十三条}\quad  《集中采购目录》经“三项工作”管委会批准后执行。采管办每年定期更新并报“三项工作”管委会批准。

\textbf{第十四条} \quad 每年初由采管办组织编制本年度采购计划,经批准后下发执行。投资项目在获得项目批复后方可列入采购计划。烟用物资要根据川渝中烟下达的生产计划,编制年度采购计划。

(一)计划提报。各职能部门于当年12月1日前提出下一年度采购需求,采购需求要包括项目名称、采购内容、采购数量、项目金额、拟采用的采购方式等,报采管办审核。拟采用非公开招标采购方式的,需提交拟采取采购方式的理由。

(二)计划审核。采管办对采购需求进行审核汇总,形成年度采购计划草案,报“三项工作”管委会审议通过,后经董事会批准。

(三)预算审核。需将讨论通过的年度采购计划提交预算管理委员会审定。

(四)计划确定。采管办根据预算管理委员会批复的年度预算,组织各职能部门调整采购计划草案,经“三项工作”管委会批准后正式形成年度采购计划,后经董事会批准。

(五)计划变更。公司必须严格按照批复的年度计划和预算实施。确因生产工作需要,需追加采购计划或变更采购内容的,要按照上述程序办理追加采购手续。已批复的采购项目,需要调整预算的要报预算管理委员会审议通过后,提交原计划审批机构审定。原则上采购计划、预算每年调整1次(公司生产经营出现大的调整除外)。

\begin{center}
\textbf{第四章\quad  采购方式及适用条件}
\end{center}

\textbf{第十五条}\quad  采购方式主要包括以下五种:

    (一)公开招标;

    (二)邀请招标;

    (三)竞争性谈判;

    (四)询价;

(五)单一来源采购;

(六)符合国家或行业规定的其他采购方式。

\textbf{第十六条}\quad  公开招标应作为主要采购方式。任何部门和个人不得将应当公开招标的采购项目化整为零或以其他任何方式规避公开招标。
国家局计划分配的烟用物资、设备及从全资三产公司采购的项目不在此规定范围内。

\textbf{第十七条}\quad  符合下列情形之一的采购,可以采用邀请招标方式采购,由集中采购部门向3个以上具备承担招标项目能力、资信良好的特定的法人或者其他组织发出投标邀请书:

    (一)技术复杂、有特殊要求或者受自然环境限制,只有少量潜在投标人可供选择;

    (二)采用公开招标方式的费用占项目合同金额的比例过大。

\textbf{ 第十八条} \quad 符合下列情形之一的,可以采用竞争性谈判、单一来源、询价采购的方式进行,但必须严格按照规定程序进行。

(一)供应商不足3家;

(二)涉及行业安全和秘密;

(三)涉及烟草行业核心技术;

(四)采用特定专利、专用技术。

\textbf{第十九条}\quad  符合下列情形之一的采购,可以采用竞争性谈判方式采购:

    (一)招标后没有供应商投标或者没有合格标的或者重新招标未能成立的;

    (二)技术复杂或者性质特殊,不能确定详细规格或者具体要求的;

    (三)采用招标所需时间不能满足用户紧急需要的;

    (四)不能事先计算出价格总额的。

  \textbf{第二十条} \quad 符合下列情形之一的采购,可以采用单一来源方式采购:

     (一)只能从唯一供应商处采购的;

    (二)发生了不可预见的紧急情况不能从其他供应商处采购的;

(三)必须保证原有采购项目一致性或者配套服务的要求,需要继续从原供应商处添购,且添购资金总额不超过原合同采购金额10\%的。

(四)政府机关、行业主管直接指定的。

\textbf{第二十一条}\quad  采购的货物规格、标准统一,现货货源充足且价格变化幅度小的,可以采用询价方式采购。

\textbf{第二十二条}\quad  国家或行业另有规定的按其规定执行。

\begin{center}
\textbf{第五章 \quad  采购程序}
\end{center}

\textbf{第二十三条} \quad 需求部门依据批复的年度采购计划、项目预算以及批准的采购方式,填写《采购项目立项审批表》,报采管办审议、“三项工作”管委会审定后实施。

\textbf{第二十四条 }\quad 招标的组织形式分为自行招标和委托招标。

\textbf{第二十五条 }\quad 建立招标代理机构库,可通过竞争性方式,一次确定多家招标代理机构,纳入招标代理机构库。

\textbf{第二十六条} \quad 建立健全评标成员管理制度,严格认定评标成员资格,加强培训考核、评价和档案管理,根据实际需求和考核情况及时对评标成员进行更换和补充,实行评标成员动态管理,评标成员库人数不少于20人。

\textbf{第二十七条}\quad  招标流程

(一)选定招标代理机构。对于招标采购项目,由相关职能部门向采管办提出申请,由采管办在招标代理机构库中随机抽取一家招标代理机构办理招标事宜。监督部门对招标代理机构抽取过程进行现场监督。

(二)招标公告发布。采用公开招标方式的,要发布招标公告。依法必须进行招标的项目,要在国务院发展改革部门依法指定的媒介发布。法律法规没有规定的,要在两家以上省级媒介发布。公司招标公告在“川渝中烟工业有限责任公司(外网)”和“烟草行业招投标信息平台”两个网站发布。

(三)标书编制。集中采购部门根据招标项目的特点和需要编制招标文件,需求部门提交技术标准及要求。招标文件要包括招标项目的技术要求、对投标人资格审查的标准、投标报价要求和评标标准等所有实质性要求和条件以及拟签订合同的主要条款。招标文件中一般要载有采用设定“最高限价”的条款以控制成本。国家对招标项目的技术、标准有规定的,要在招标文件中提出相应要求。招标文件不得要求或者标明特定的生产供应者以及含有倾向或者排斥潜在投标人的其他内容。

(四)审定标书。由集中采购部门填写《采购项目实施审批表》,报采管办,采管办组织审议招标文件,经“三项工作”管委会讨论通过后方可发布。

(五)投标人资质审查。投标人应当提供有关资质证明和业绩情况。采管办要组织相关职能部门人员根据采购项目的具体特点和实际需要,对投标人资质进行审查。资质审查未通过的供应商不得进入采购环节。

(六)开标。开标要在招标文件确定的时间内并在预先确定的地点公开进行。开标全过程要有监督部门的代表进行现场监督,重大采购项目可以同时聘请公证机构进行公证。

(七)评标。评标工作由评标委员会负责,评标委员会独立履行下列职责:审查投标文件是否符合招标文件要求并作出评价;要求投标供应商对投标文件有关事项作出解释或者澄清;推荐中标候选供应商名单;向“三项工作”管委会报告非法干预评标工作的行为。

自行招标的评标委员会由相关职能部门代表和有关技术、经济等方面的专家及专业技术人员组成,成员人数须为5人以上单数,其中技术、经济等方面的专家及技术人员不得少于成员总数的2/3,其人选由采管办会同相关职能部门在监督部门现场监督下从评标成员库中随机抽取。评标全过程需由纪检和审计部门以及职工代表进行现场监督。

(八)采管办要在评标结束后3个工作日内,将评标报告提交“三项工作”管委会,待会议审定后公布。

(九)招标结果需在一定范围内进行公开,包括中标单位、中标数量、中标金额等内容,公示时间不少于3个工作日。

\textbf{第二十八条 }\quad 有下列情形之一的,评标委员会应当否决其投标:

(一)投标文件未经投标单位盖章和单位负责人签字;

    (二)投标联合体没有提交共同投标协议;

    (三)投标人不符合国家或者招标文件规定的资格条件;

    (四)同一投标人提交两个以上不同的投标文件或者投标报价,但招标文件要求提交备选投标的除外;

    (五)投标报价低于成本或者高于招标文件设定的最高投标限价;

    (六)投标文件没有对招标文件实质性要求和条件作出响应;

    (七)投标人有串通投标、弄虚作假、行贿等违法行为。

\textbf{第二十九条}\quad  否决投标后,除采购项目取消情形外,要重新组织招标;需要采用其他采购方式的,要在采购活动开始前经“三项工作”管委会批准。

\textbf{第三十条} \quad 采用竞争性谈判方式采购的,要遵循以下程序:

    (一)成立谈判小组。由集中采购部门牵头组织相关职能部门及有关专家组成谈判小组,谈判小组由3人以上的单数组成,其中相关职能部门的代表不能超过成员总数的1/3。

    (二)集中采购部门负责组织谈判小组制订谈判文件,谈判文件应当明确谈判程序、谈判内容、合同草案条款以及评定成交的标准等事项。谈判文件需交采管办组织审核,并最终由“三项工作”管委会审定。

    (三)谈判小组确定参加谈判的供应商。需从符合资格条件的供应商名单中选取不少于3家符合相应资格条件的供应商参加谈判并向其提供谈判文件。

    (四)谈判小组全体成员应当在监督工作小组成员监督下集中与单一供应商分别进行谈判。

    (五)确定成交供应商。谈判小组根据符合采购需求、质量和服务相等且报价最低的原则确定成交供应商,向采管办提交采购意见,由相关职能部门填写《采购项目实施过程记录表》,经采管办审议、“三项工作”管委会审定后,将结果通知所有参加谈判的供应商。


\textbf{第三十一条}\quad  采用询价方式采购的,要遵循以下程序:
    (一)成立询价小组。由集中采购部门牵头组织相关职能部门及有关专家组成询价小组,询价小组由3人以上的单数组成,其中相关职能部门的代表不能超过成员总数的1/3。

(二)询价小组制订询价文件,询价文件应当明确询价程序、询价内容、合同草案条款以及评定成交的标准等事项。询价文件需由采管办审议、“三项工作”管委会审定。

(三)询价小组要对采购项目的价格构成和评定成交的标准等事项作出规定。

(四)确定被询价的供应商名单。询价小组根据采购需求,从符合相应资格条件的供应商名单中确定不少于3家的供应商并向其发出询价通知书让其报价。

(五)询价。询价小组要求被询价的供应商一次报出不得更改的价格。

(六)确定成交供应商。询价小组根据符合采购需求、质量和服务相等且报价最低的原则确定成交供应商并向采管办提交采购意见,由相关职能部门填写《采购项目实施过程记录表》,经采管办审议、“三项工作”管委会审定后,将结果通知所有参加询价的供应商。

\textbf{第三十二条} \quad 采用单一来源方式采购的,要遵循以下程序:

由集中采购部门组织相关职能部门、财务、审计、法规、监察等部门人员,在保证采购货物、工程和服务质量前提下,与供应商商定合理价格并形成采购意见,由相关职能部门填写《采购项目实施过程记录表》,报采管办审议、“三项工作”管委会审定。未通过的采购项目,根据采管办意见与供应商重新协商或取消采购。

\textbf{第三十三条} \quad 质疑投诉受理。投标人或供应商对招标结果有异议的,应当在结果发布之日起7个工作日内,以书面形式向公司监察部门提出质疑。监察部门在收到投标人或供应商书面质疑后,要对质疑投诉进行调查,作出处理。对重大投诉事项,监察部门要报“三项工作”管委会研究后提出处理意见。

\textbf{第三十四条 }\quad 项目变更。根据工作需要,确需变更项目的,要按照“谁审定、谁负责”的原则,报原审定机构审议。变更的具体内容要由需求部门提报详细的书面材料报采管办,采管办审核并提出具体意见后经原审批机构审议。

\textbf{第三十五条}\quad  零星采购行为管理。难以与公司《采购目录》规定的采购项目进行合理打包,且采购金额在2万元以下的采购行为,可以作为零星采购行为进行管理,并按照公司报销相关工作制度执行。

\textbf{第三十六条}\quad  紧急采购行为管理。公司在面临抗灾抢险、疫情紧迫、安全事故等不可预见、无法避免的情况下,可以采取紧急采购手段完成采购行为。紧急采购执行部门填写《紧急采购行为审核表》提报公司采购管理办公室进行审核通过后,按照公司报销相关工作制度执行。待造成紧急采购行为的原因结束后15日内,紧急采购部门将紧急采购原因、采购品种规格、采购金额、验收、支付等相关情况提报公司董事会进行备案审核。

\begin{center}
\textbf{第六章 \quad  供应商管理}
\end{center}

\textbf{第三十七条 }\quad 集中采购部门(供应科)需加强对供应商的日常管理,制定《供应商管理规范》。建立供应商资质认证制度、供应商动态评价和退出机制,制订供应商资质认定标准,通过综合评价建立合格供应商名录。加强对长期的供应商动态管理,定期对产品和服务质量等进行跟踪评审,依据评审意见,对评价不合格的供应商,取消其供应资格。

\textbf{第三十八条} \quad  供应商参加采购活动时应当具备下列条件:

(一)具有独立承担民事责任的能力。

(二)具有良好的商业信誉和健全的财务会计制度。

(三)具有履行合同所必需的设备和专业技术能力。

(四)有依法缴纳税收和社会保障资金的良好记录。

(五)参加采购活动前3年内,在经营活动中没有重大违法记录。

(六)在劳动保护、节能减排与生态环境保护方面符合国家规定要求。

(七)法定代表人不能参加采购活动时,可委托他人参加,但需提供授权委托书。

(八)法律、行政法规规定的其他条件。

\textbf{第三十九条}\quad  根据采购项目的特殊要求,可以规定供应商的特定条件,但不得以不合理的条件对供应商实行差别待遇或者歧视待遇。

\textbf{第四十条}\quad  烟用物资采购项目中,对总公司发布的烟用物资供应商名录,三联公司须从名录中通过招标等方式选择供应商。

\begin{center}
\textbf{第七章 \quad  签约和履约}
\end{center}

\textbf{第四十一条 }\quad 集中采购部门负责起草采购合同,填写《合同会签审批表》,需经立项部门、财务审计科、政工科(纪检)、法律顾问、整顿办、分管领导审核通过后,由企业法定代表人与中选供应商签订合同;非法定代表人要取得法定代表人的授权后与中选供应商签订合同。

招标采购合同应当在自中标通知书发出之日起30日内与成交供应商签订。

\textbf{第四十二条}\quad  采购合同签订后,在合同有效期内由集中采购部门依据合同约定,按照资金审批程序办理款项支付手续。

\textbf{第四十三条}\quad  相关职能部门要及时对供应商履约情况进行验查并登记造册,办理入库或获得手续,建立相应的实物和固定资产台账。大型或者复杂的采购项目,要邀请有相应资质的质量检测机构参加验收工作。所有验收人员要在验收书上签字并承当相应的责任。

工程类项目竣工验收决算必须经过审计后,项目审批单位才可组织竣工验收。

烟用物资类项目由质检部门负责依据质量标准对采购物资实施质量检验与验证并出具报告。质检人员及其负责人要在报告上签字并承担相应责任。

\begin{center}
\textbf{第八章 \quad  采购档案管理}
\end{center}

\textbf{第四十四条}\quad  集中采购部门要根据工作制度和实施流程建立五种采购方式的“一项一卷”归档标准,妥善保管相关资料。

\textbf{第四十五条}\quad  招标采购归档资料要至少包括:采购计划及预算、立项会议纪要、采购事项审批表、招标文件、投标文件、评标标准、评估报告、定标文件、评委评分表、质疑答复、投诉处理决定、确定供应商依据、合同文本及法人代表授权委托书、验收证明、采购活动记录及其他有关文件、资料。

\textbf{第四十六条 }\quad 竞争性谈判归档资料要至少包括:采购计划及预算、立项会议纪要、采购事项审批表、采购方式确定依据、谈判人员确定依据、谈判文件、确定供应商依据、合同文本及法人代表授权委托书、验收证明、采购活动记录及其他有关文件、报表、资料。

\textbf{第四十七条}\quad  单一来源归档资料要至少包括:采购计划及预算、立项会议纪要、采购事项审批表、采购方式确定依据、价格协商相关记录、确定供应商依据、合同文本、验收证明、采购活动记录及其他有关文件、报表、资料。

\textbf{第四十八条}\quad  询价归档资料要至少包括:采购计划及预算、立项会议纪要、采购事项审批表、采购方式确定依据、询价过程相关资料、确定供应商依据、合同文本、验收证明、采购活动记录及其他有关文件、报表、资料。

\begin{center}
\textbf{第九章 \quad  监督和责任}
\end{center}

\textbf{第四十九条} \quad 建立严格的包含采购活动工作程序、职责、内部监督的工作制度,经办采购的人员与合同审核、验收、付款人员的职责需明确并相互分离。

\textbf{第五十条}\quad  法规部门(整顿办、政工科外聘律师)负责采购合同的合法性审核及采购活动中有关法律法规问题的咨询和指导。财务部门负责采购的资金管理工作。审计职能部门负责对采购活动实施程序性、全过程跟踪或实质性审计。

\textbf{第五十一条}\quad  采购活动由纪检监察职能部门按照《烟草行业投标采购活动廉政监督工作暂行规定(试行)》(国烟监[2008]~27号)进行监督,对违反相关规定的行为,将依法依规进行处理。

\begin{center}
\textbf{第十章 \quad  附  则}
\end{center}

\textbf{第五十二条}\quad 本规定由三联公司采管办负责解释。

\textbf{第五十三条}\quad 本规定自印发之日起施行。

