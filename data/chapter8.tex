\chapter{易地技改专题报告}
\section{易地技改专题报告}
三联公司易地技改项目于2013年1月7日得到中国烟草总公司批复(中烟办[2013]9号),于2013年1月28日收到川渝中烟工业有限责任公司转发批文通知(川渝烟工基建[2013]43号)。根据通知要求,三联公司开展了以下几项工作:

一、组建项目建设组织机构,草拟机构人员名单,于2月28日书面上报川渝公司基建技改办公室,待川渝公司批示。

二、多次积极与龙泉区政府、投资部、生产部沟通,如何快速推进项目实施,邀请龙泉区政府相关负责人座谈,希望尽快办理土地购置。根据龙泉经开区关于诚邀川渝中烟工业公司落户经开区若干优惠政策的承诺函,于5月20日与经开区投资服务局签订项目建设协议。

三、8-10月根据龙泉经开区要求,请中国轻工业成都设计工程有限公司绘制项目建设效果图,经投资部同意后交经开区审批,经反复沟通修改,得到经开区口头同意,报国土局准备挂牌。10月下旬经与经开区沟通协调,并提交土地挂牌申请报告到建设服务局,沟通挂牌条件,准备挂牌。

四、2013年11月上旬川渝明确由投资部牵头,三联公司配合土地摘牌。2013年11月15日投资部通知项目建设用地调整为170亩摘牌,并配合提交相关手续;12月10日投资部通知项目建设用地调整为160亩摘牌,并配合更改相关手续。

五、2014年3月初,专题向投资部汇报:与龙泉区政府沟通土地挂牌事项遇到的困难,投资部领导与龙泉区政府领导沟通后答复:等待龙泉区政府土地协调会的结果。

在三联公司易地技改工作推进中,由于项目建设指挥部未成立,项目设计规划及相关工作无法进行。经开区认为我公司无项目设计方案,相关土地挂牌摘牌工作也不积极,能拖则拖,项目建设推动迟缓。为加快项目建设,当前急需完成以下工作:

(一)尽快成立项目建设指挥部推动项目建设,制定行动计划,包括变更净地由200亩变更为160亩后的投资协议。

(二)项目建设指挥部成立后,抓紧进行项目设计规划,明确易地技改的重点和要突出的亮点,描绘技改蓝图,进而组织设计招标;同时进行项目制度建设,规范项目建设过程中的各项工作操作流程和标准;财务人员准备技改筹资计划。

(三)设计招标完成后,及时沟通做好项目设计方案,报国家局审批,并跟踪审批进度;同时将项目设计方案与经开区沟通,促进土地挂牌摘牌,并办理报建手续。

(四)设计方案经国家局批复,组织项目建设招标、项目管理招标、项目监理招标等准备工作,项目进入实质开工建设阶段。
需川渝公司给予支持的工作:
批准成立项目建设指挥部;协调土地摘牌;与国家局计划司协调审批进度;财务支持;项目建设管理支持。


