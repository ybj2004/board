\chapter{四川三联卷烟材料有限公司章程}\label{app:appendix0-1}
\section{四川三联卷烟材料有限公司章程}
\begin{center}
\textbf{第一章 \quad  总  则}
\end{center}

\textbf{第一条 }\quad 公司宗旨:通过有限责任公司组织形式,由股东共同出资,筹集资本金,建立新的经营机制,为川渝卷烟工业的快速发展作出贡献。依照《中华人民共和国公司法》和《中华人民共和国公司登记管理条例》的规定,制定本公司章程。

\textbf{第二条} \quad 公司名称:四川三联卷烟材料有限公司。(以下简称公司)

\textbf{第三条} \quad 公司住所:成都高新区桂溪乡和平村。

\textbf{第四条} \quad 公司由3个法人股东共同出资设立。股东以其认缴出资额为限对公司承担责任;公司以其全部资产对公司的债务承担责任。公司享有由股东投资形成的全部法人财产权,依法享有民事权利,承担民事责任,具有企业法人资格。

\textbf{第五条} \quad 经营范围:生产、经营卷烟滤咀棒及有关配套产品。

\textbf{第六条} \quad 公司营业执照签发之日,为本公司成立之日。营业期限:1996年4月5日至2016年4月4日。


\begin{center}
\textbf{第二章 \quad  注册资本}
\end{center}

\textbf{第七条}\quad  公司注册资本为8000万元人民币,公司实收资本为8000万元人民币。公司注册资本为在公司登记机关依法登记的全体股东认缴的出资额,实收资本为全体股东实际交付并经公司登记机关依法登记的出资额。

\textbf{第八条}\quad  股东名称、出资额、出资方式、出资时间一览表。

 \begin{table}[!htbp]
    \renewcommand{\arraystretch}{1.3}
    \centering
     \begin{tabular}
  {>{\sf }p{0.35\textwidth}<{\centering}p{0.2\textwidth}<{\centering}p{0.15\textwidth}<{\centering}p{0.2\textwidth}<{\centering}}
   \toprule[1pt]
  \sf 股东名称 & \sf 出资额(万元) & \sf 出资方式 & \sf 出资时间 \\
\midrule
川渝中烟工业有限责任公司 & 6080& 货币& 2002.7.16  \\
四川烟草工业有限责任公司 & 1680 & 货币 & 2002.7.16  \\
重庆烟草工业有限责任公司 & 240 & 货币 & 2002.7.16  \\
\bottomrule[1pt]
    \end{tabular}
    \end{table}




\textbf{第九条} \quad 公司登记注册后,应向股东签发出资证明书。出资证明书应载明公司名称、公司成立日期、公司注册资本、股东的姓名或者名称、缴纳的出资额和出资日期、出资证明书的编号和核发日期。出资证明书由公司盖章。出资证明书一式两份,股东和公司各持一份。
出资证明书遗失,应立即向公司申报注销,经公司董事会审核同意予以补发。

\textbf{第十条}\quad  公司应设置股东名册,记载股东的姓名、住所、出资额及出资证明书编号。


\begin{center}
\textbf{第三章 \quad  股东的权利、义务和转让出资的条件}
\end{center}

\textbf{第十一条} \quad 股东作为出资者按投入公司的资本额,享有资产受益、重大决策和选择管理者等权利,并承担相应的义务。

\textbf{第十二条} \quad 股东的权利:

    一、出席股东会,并根据其出资额享有表决权;

    二、股东有权查阅股东会会议记录和公司财务会计报告;

    三、选举和被选举为董事会成员、监事会成员;

    四、股东按出资比例分取红利。公司新增资本时,股东可按出资比例优先认缴出资;

    五、优先购买其他股东转让的出资;

六、查阅、复制公司章程、股东会议记录、董事会决议、监事会决议和财务报告。

七、公司终止后,依法分取公司的剩余财产。

\textbf{第十三条} \quad 股东义务:

    一、按期足额缴纳所认缴的出资;

    二、依其所认缴的出资额承担公司债务;

    三、公司办理工商登记注册后,不得抽回出资,违者应赔偿其他股东因此而遭受的损失;

四、遵守公司章程规定的各项条款。

\textbf{ 第十四条} \quad 转让出资的条件:

    一、股东之间可以相互转让其全部出资或者部分出资。

    二、股东向股东以外的人转让股权的,必须经其他股东过半数同意。股东应就其股权转让事项书面通知其他股东征求同意,其他股东自接到书面通知之日起满三十日未答复的,视为同意转让。其他股东半数以上不同意的,不同意转让的股东应当购买该转让的股权;不购买的,视为同意转让。

    三、经股东同意转让的股权,在同等条件下,其他股东对该出资有优先购买权。两个以上股东主张行使优先购买权的,协商确定各自的购买比例;协商不成的,按照转让时各自的出资比例行使优先购买权。

    四、股东依法转让其出资后,由公司将受让人的姓名或者名称、住所以及受让的出资额记载于股东名册。


\begin{center}
\textbf{第四章\quad  公司的机构及高级管理人员的资格和义务}
\end{center}

\textbf{第十五条 }\quad 为保障公司生产经营活动的顺利、正常开展,公司设立股东会、董事会和监事会,负责全公司生产经营活动的预测、决策和组织领导、协调、监督等工作。

\textbf{第十六条} \quad 本公司设总经理、生产经营科、财务审计科等具体办理机构,分别负责处理公司在开展生产经营活动中的各项日常具体事务。

\textbf{第十七条} \quad 董事、监事、经理应遵守公司章程、《中华人民共和国公司法》和国家其他有关法规的规定。

\textbf{第十八条}\quad  公司研究决定有关职工工资、福利、安全生产以及劳动保护、劳动保险等涉及职工切身利益的问题,应当事先听取公司工会和职工的意见,并邀请工会或者职工代表列席有关会议。

\textbf{第十九条} \quad 公司研究决定生产经营的重大问题、制定重要的规章制度时,应当听取公司工会和职工的意见和建议。

\textbf{第二十条} \quad 有下列情形之一的人员,不得担任公司董事、监事、经理:

(一)	无民事行为能力或者限制民事行为能力者;

(二)	因犯有贪污、贿赂、侵占财产、挪用财产罪或者破坏社会经济秩序罪;被判处刑罚,执行期未满逾五年,或者因犯罪被剥夺政治权利。执行期满未逾五年者;

(三)	担任因经营不善破产清算公司(企业)的董事或者厂长、经理,并对该公司(企业)破产负有个人责任的,自该公司(企业)破产清算完结之日起未逾三年者;

(四)	担任因违法被吊销营业执照的公司(企业)的法定代表人,并负有个人责任的,自该公司(企业)被吊销营业执照之日未逾三年者;

(五)	个人所负数额较大的债务到期未清者。

公司违反前款规定选举董事、监事或者聘任经理的,该选举或者聘任无效。

\textbf{第二十一条} \quad 国家公务员不得兼任公司的董事、监事、经理。

\textbf{第二十二条}\quad  董事、监事、经理应当遵守公司章程,忠实履行职责,维护公司利益,不得利用在公司的地位和职权为自己谋取私利。董事、监事、经理不得利用职权收受贿赂或者其他非法收入,不得侵占公司的财产。

\textbf{第二十三条}\quad  董事、经理不得挪用公司资金或者将公司资金借给任何与公司业务无关的单位和个人。

董事、经理不得将公司的资金以其个人名义或者以其他个人名义开立帐户存储,亦不得将公司的资金以个人名义向外单位投资。

董事、经理不得以公司资产为本公司的股东或者其他个人债务提供担保。

\textbf{第二十四条}\quad  董事、经理不得自营或者为他人经营与其所任职公司经营相同或相近的项目,或者从事损害本公司利益的活动。从事上述营业或者活动的,所得收入应当归公司所有。


\begin{center}
\textbf{第五章 \quad  股 东 会}
\end{center}

\textbf{第二十五条} \quad 公司设股东会,公司股东会由全体股东组成,为公司的最高权力机构。股东会会议,由股东按照出资比例行使表决权。出席股东会的股东必须超过全体股东表决权的半数以上方能召开。首次股东会由出资最多的股东主持,以后股东会由董事会召集、董事长主持。

\textbf{第二十六条}  \quad 股东会行使以下职权:

    1.决定公司的经营方针和投资计划;

    2.选举和更换非由职工代表担任的董事、监事,决定有关董事、监事的报酬事项;

    3.审议批准董事会的报告,监事会或监事的报告;

    4.审议批准公司年度财务预算方案、决算方案和利润分配方案、弥补亏损方案;

5.对公司增加或减少注册资本作出决议;

6.对公司的合并、分立、解散、清算或者变更公司形式作出决议;

    7.对发行公司债券作出决议;

 8.修改公司章程。
 
 股东会分定期会议和临时会议。股东会每年定期召开,由董事会召集,董事长主持。董事长不能履行或者不履行主持股东会会议职责的,由半数以上董事共同推举一名董事主持。
 
董事会不能履行召集股东会会议职责的,由监事会召集和主持;监事会不召集和主持的,代表十分之一以上表决权的股东可以自行召集和主持。召开股东会会议,应于会议召开十五日前通知全体股东。

(一)股东会议应对所议事项作出决议。对于修改公司章程、增加或减少注册资本、分立、合并、解散或变更公司形式等事项作出的决议,必须经代表三分之二以上表决权的股东同意通过;

(二)股东会应对所议事项作成会议记录,出席会议的股东应在会议记录上签名,会议记录作为公司档案材料长期保存;

(三)对前款所列事项股东以书面形式一致表示同意的,可以不召开股东会会议,直接作出决议,并由全体股东在决议文件上签名、盖章。


\begin{center}
\textbf{第六章 \quad  董事会、经理、监事会}
\end{center}

\textbf{第二十七条}\quad 本公司设董事会,董事会是公司的执行机构。公司董事会由5名董事组成。其中,股东董事由股东会代表公司股权过半数股东同意选举产生,共4名,职工董事由职工代表大会、职代会或者其他民主形式民主选举,共1名。

\textbf{第二十八条}\quad 董事长为公司法定代表人。董事长由公司三分之二以上的董事选举产生。

\textbf{第二十九条}\quad 董事会对股东会负责,行使以下权利:

    一、负责召集股东会,并向股东会报告工作;

    二、执行股东会的决议;

    三、决定公司的经营计划和投资方案;

四、制订公司年度财务预、决算方案;

五、制订公司的利润分配方案和弥补亏损方案;

    六、制订公司增加或减少注册资本、合并、分立、解散、变更公司形式的方案;

    七、决定公司内部管理机构的设置;

    八、决定聘任或者解聘公司经理及其报酬事项,并根据经理的提名,决定聘任或者解聘公司副经理、财务负责人及其报酬事项;

九、制定公司的基本管理制度;

十、公司章程规定的其他职权。

\textbf{第三十条}\quad 董事任期为三年,可以连选连任。
     董事会会议由董事长召集和主持,董事长因特殊原因不能履行职务或不履行职务时,由半数以上董事共同推举一名董事召集和主持。
     召开董事会会议,应当于会议召开十日以前通知全体董事。三分之一以上董事可以提议召开董事会会议。董事会会议决议,实行一人一票。
     董事会对所议事项的决定作成会议记录,出席会议的董事应在会议记录上签名。

\textbf{第三十一条}\quad 公司经理由董事会聘任或者解聘。经理对董事会负责,负责公司日常经营管理工作,行使以下职权:

    一、主持公司的生产经营管理工作,组织实施董事会决议;

    二、组织实施公司年度经营计划和投资方案;

    三、拟订公司内部管理机构设置的方案;

    四、拟订公司基本管理制度;

    五、制定公司的具体规章;

    六、提请聘任或者解聘公司副经理、财务负责人;

七、决定聘任或者解聘除应由董事会聘任或者解聘以外的负责管理人员;

八、董事会授予的其他职权。

经理列席董事会议。

\textbf{第三十二条}\quad 董事、监事、公司经理应遵守公司章程和《公司法》的有关规定。

\textbf{第三十三条}\quad 公司设立监事会,是公司的监督机构。其成员由股东会代表公司二分之一以上表决权的股东选举产生,公司监事会由 3名监事组成,其中股东代表2名,公司职工代表1名。

监事会主席由公司监事过半数选举产生。监事任期为每届三年,届满可连选连任。

监事任期届满未及时改选,或者监事在任期内辞职导致监事会成员低于法定人数的,在改选出的监事就任前,原监事仍应当依照法律、行政法规和公司章程的规定,履行监事职务。

监事可以列席董事会议,并对董事会决议事项提出质询或者建议。

监事会每年度至少召开一次会议,监事可以提议召开临时监事会会议。监事会决议应当经全体监事半数以上通过。监事会应当对所议事项的决定作成会议记录,出席会议的监事应当在会议记录上签名。

    监事会的职权:

(一)检查公司财务;

(二)对董事、高级管理人员执行公司职务的行为进行监督,对违反法律、行政法规、公司章程或者股东会决议的董事、高级管理人员提出罢免的建议;

(三)当董事和经理的行为损害公司的利益时,要求董事和经理予以纠正;在董事不履行本法规定的召集和主持股东会会议职责时召集和主持股东会会议;

(四)向股东会会议提出提案;

(五)依照《公司法》第一百五十二条的规定,对董事、高级管理人员提起诉讼;

(六)公司章程规定的其他职权。


\begin{center}
\textbf{第七章 \quad  财务、会计}
\end{center}

\textbf{第三十四条} \quad 公司依照法律、行政法规和国家财政行政主管部门的规定建立本公司的财务、会计制度。

\textbf{第三十五条}\quad  公司在每一会计制度终了时制作财务会计报表,按国家和有关部门的规定进行审计并出具审计报告,送交各股东审查。

\textbf{第三十六条 } \quad 公司分配每年税后利润时,提取利润的百分之十列入法定公积金,公司法定公积金累计额超过公司注册资本百分之五十时可不再提取。公司的公积金用于弥补以前年度公司的亏损、扩大公司生产经营或者转为增加公司资本。但是,资本公积金不得用于弥补公司的亏损。

\textbf{第三十七条} \quad  公司弥补亏损和提取公积金后所余税后利润,按照股东出资比例进行分配。

\textbf{第三十八条} \quad 法定公积金转为资本时,所留存的该项公积金不得少于转增前公司注册资本的百分之二十五。

公司除法定会计帐册外,不得另立会计帐册。

会计帐册、报表及各种凭证应按财政部有关规定装订成册归档,作为重要的档案资料妥善保管。


\begin{center}
\textbf{第八章 \quad   合并、分立和变更注册资本}
\end{center}

\textbf{第三十九条} \quad  公司合并或者分立,由公司的股东会作出决议;按《公司法》的要求签订协议,清算资产、编制资产负债及财产清单,通知债权人并公告,依法办理有关手续。

\textbf{第四十条}\quad  公司合并、分立、减少注册资本时,应编制资产负债表及财产清单。公司应当自作出合并分立决议之日起10内通知债权人,并于30日内在报纸上公告。债权人自接到通知书之日起30日内,未接到通知书的自公告之日起45日内,有权要求公司清偿债务或提供相应担保。公司分立前的债权债务由分立后的公司承担连带责任。

\textbf{第四十一条}\quad  公司合并或者分立,登记事项发生变更的,应当依法向公司登记机关办理变更登记;公司解散的,应当依法办理公司注销登记;设立新公司的,应当依法办理公司设立登记。

公司增加或减少注册资本,应当依法向公司登记机关办理变更登记。


\begin{center}
\textbf{第九章 \quad  破产、解散、终止和清算}
\end{center}

\textbf{第四十二条}\quad   公司因《公司法》第181 条所列(1)(2)(4)(5)项规定而解散时,应当在解散事由出现之日起15日内成立清算组,开始清算。逾期不成立清算组进行清算的,债权人可以申请人民法院指定有关人员组成清算组进行清算。

公司清算组应当自成立之日起10日内通告债权人,并于60日内在报纸上公告。债权人应当自接到通知书之日起30日内,未接到通知书的自公告之日45日内,向清算组申报债权。

公司财产在分别支付清算费用、职工的工资、社会保险费用和法定补偿金,交纳所欠税款,清偿公司债务后的剩余资产,有限责任公司按照股东的出资比例分配,股份有限公司按照股东持有的股份比例分配。

公司清算结束后,公司应当依法向公司登记机关申请注销公司登记。

\begin{center}
\textbf{第十章 \quad 工会}
\end{center}

\textbf{第四十三条}\quad   公司按照国家有关法律和《中华人民共和国工会法》设立工会。工会独立自主地开展工作,公司应支持工会的工作。公司劳动用工制度严格按照《劳动法》执行。


\begin{center}
\textbf{第十一章 \quad  附  则}
\end{center}

\textbf{第四十四条}\quad  公司章程的解释权属公司股东会。

\textbf{第四十五条}\quad  公司章程经全体股东签字盖章生效。

\textbf{第四十六条}\quad   经股东会提议公司可以修改章程,修改章程须经股东会代表公司三分之二以上表决权的股东通过后,由公司法定代表人签署并报公司登记机关备案。

\textbf{第四十七条}\quad  因本章程产生的或与本章程有关的争议,选择下列第(一)种方式解决:

(一)提交成都仲裁委员会仲裁;

   (二)依法向人民法院起诉。

\textbf{第四十八条 }\quad 公司章程与国家法律、行政法规、国务院决定等有抵触,以国家法律、行政法规、国务院决定等为准。

\vskip 2cm
 \rightline{法人股东盖章\qquad}

