\chapter{公司章程修正案}
\section{四川三联卷烟材料有限公司章程修正案}\sanhao
国家最新《公司法》于2014年3 月1日起正式施行。
按照新的公司法的规定和要求,我公司章程有\textbf{四处}需要修订。现将需要修订的条款说明如下(公司新章程全文见附录~\ref{app:appendix0-1}):



%\tikzset{
%>=stealth',
%  punktchain/.style={
%    rectangle,
%    rounded corners,
%    % fill=black!10,
%    draw=black, very thick,
%    text width=3cm,
%    minimum height=3em,
%    text centered,
%    on chain},
%%  line/.style={draw, thick, <-},
%   every join/.style={->, thick,shorten >=1pt},
%%  decoration={brace},
%}
%
%\begin{table}[htbp]
%\dawuhao \song
%   \renewcommand{\arraystretch}{1.6}
%   \centering
%   \caption{公司章程修正说明表}
% \begin{tabular}
%   {|>{\sf}p{0.125\textwidth}|p{0.4\textwidth}|p{0.4\textwidth}|}
%  \hline
%  \rowcolor{darkblue!20}
%   \sf 条款   &  \sf 旧章程 & \sf 新章程   \\
%  \hline \hline
%第九条&各股东认缴、实缴的公司注册资本应在申请公司登记前,委托会计师事务所进行验证&此条删除\\
%  \hline
%  第二十六条&
%\begin{minipage}[c]{1.0\linewidth}
%%会议主持人的补缺次序
%\centering
% \begin{tikzpicture}
%  [node distance=.6cm,
%  start chain=going below,]
%     \node[punktchain, join] (a) {董事长};
%     \node[punktchain, join] (c) {副董事长};
%     \node[punktchain, join] (d) {监事会};
%     \node[punktchain, join] (e) {股东(代表十分之一以上表决权)};
%  \end{tikzpicture}
%  \end{minipage}
%%%%%
%&
%%%%%
%\begin{minipage}[c]{1.0\linewidth}
%\centering
% \begin{tikzpicture}
%  [node distance=.6cm,
%  start chain=going below,]
%     \node[punktchain, join] (aa) {董事长};
%     \node[punktchain, join] (cc) {董事(半数以上董事共同推举)};
%     \node[punktchain, join] (dd) {监事会};
%     \node[punktchain, join] (ee) {股东(代表十分之一以上表决权)};
%  \end{tikzpicture}
%  \end{minipage}\\
%   \hline
%第四十五条&公司章程经{\hei{法定代表人}}签字盖章生效&公司章程经{\hei{全体股东}}签字盖章生效\\
%   \hline
% 第四十七条&& \begin{minipage}[t]{1.0\linewidth}
%因本章程产生的或与本章程有关的争议,选择下列第(一)种方式解决:\\
%(一)提交成都仲裁委员会仲裁;\\
%(二)依法向人民法院起诉。\\
% \end{minipage}\\
%   \hline
%\end{tabular}
% \end{table}



\begin{table}[htbp]
\dawuhao \song
   \renewcommand{\arraystretch}{1.6}
   \centering
   \caption{公司章程修正说明表}
 \begin{tabular}
   {|>{\sf}p{0.125\textwidth}|p{0.4\textwidth}|p{0.4\textwidth}|}
  \hline
  \rowcolor{darkblue!20}
   \sf 条款   &  \sf 旧章程 & \sf 新章程   \\
  \hline \hline
第九条&各股东认缴、实缴的公司注册资本应在申请公司登记前,委托会计师事务所进行验证&此条删除\\
  \hline
  第二十六条&董事长不能履行或者不履行召集股东会会议职责的,{\hei{由副董事长召集和主持}};副董事长不能或者不履行召集和主持股东会议职责的,由监事会召集和主持;监事会不召集和主持的,代表十分之一以上表决权的股东可以自行召集和主持。
          &董事长不能或者不履行主持股东会议职责的,{\hei{由半数以上董事共同推举一名董事主持。}}
董事会不能履行召集股东会会议职责的,由监事会(不设监事会的由监事)召集和主持;监事会或者监事不召集和主持的,代表十分之一以上表决权的股东可以自行召集和主持。召开股东会会议,应于会议召开十五日前通知全体股东(注:具体通知时间可由公司章程自定)\\
   \hline
第四十五条&公司章程经{\hei{法定代表人}}签字盖章生效&公司章程经{\hei{全体股东}}签字盖章生效\\
   \hline
 第四十七条&& \begin{minipage}[t]{1.0\linewidth}
因本章程产生的或与本章程有关的争议,选择下列第(一)种方式解决:\\
(一)提交成都仲裁委员会仲裁;\\
(二)依法向人民法院起诉。\\
 \end{minipage}\\
   \hline
\end{tabular}
 \end{table}


%{\dawuhao \song {  %此处写字体大小控制命令
%\begin{center}
%  \renewcommand{\arraystretch}{1.5}
%\begin{longtable}{|p{0.125\textwidth}|p{0.4\textwidth}|p{0.4\textwidth}|}
% \caption{公司章程修正说明表}\\
%\hline
%\hline
%\rowcolor{darkblue!20}
% \sf 条款   &  \sf 旧章程 & \sf 新章程   \\
%\hline
%\endfirsthead    % endfirsthead      %%%需要定义两个头两个尾的样式
%%
%\multicolumn{3}{c}{\bf{公司章程修正说明表(续表)}}\\
%\hline
%\rowcolor{lightblue}
%条款   &  \sf 旧章程 & \sf 新章程  \rule{0pt}{20pt} \\
%\hline
%\endhead         % endhead
%%
%%%%%
%\hline
%\multicolumn{3}{r}{\emph{ 续下页 \ldots}}
%\endfoot          % endfistfoot
%%
%\hline
%\endlastfoot       % endfoot
%  \hline
%第九条&各股东认缴、实缴的公司注册资本应在申请公司登记前,委托会计师事务所进行验证&此条删除\\
%  \hline
%  第二十六条&副董事长不能或者不履行召集和主持股东会议职责的,\textbf{由监事会召集和主持};监事会不召集和主持的,代表十分之一以上表决权的股东可以自行召集和主持。&副董事长不能或者不履行主持股东会议职责的,\textbf{由半数以上董事共同推举一名董事主持。}
%董事会不能履行召集股东会会议职责的,由监事会(不设监事会的由监事)召集和主持;监事会或者监事不召集和主持的,代表十分之一以上表决权的股东可以自行召集和主持。召开股东会会议,应于会议召开十五日前通知全体股东(注:具体通知时间可由公司章程自定)\\
%   \hline
%第四十五条&公司章程经\textbf{法定代表人}签字盖章生效&公司章程经\textbf{全体股东}签字盖章生效\\
%   \hline
% 第四十七条&& \begin{minipage}{1.0\linewidth}
%因本章程产生的或与本章程有关的争议,选择下列第(一)种方式解决:\\
%(一)提交成都仲裁委员会仲裁;\\
%(二)依法向人民法院起诉。
% \end{minipage}\\
%   \hline
%\end{longtable}
%\end{center}
%}}

%\begin{center}
%\textbf{ 一、第九条:}
%\end{center}
%
%
%旧章程 \hfill 新章程\\
%\fbox{\parbox[t][5cm]{0.4\textwidth}
%{各股东认缴、实缴的公司注册资本应在申请公司登记前,委托会计师事务所进行验证。
%}}
%%
%\hfill
%%$\HandRight$
%\fbox{\parbox[t][5cm]{0.4\textwidth}
%{新章程中, 将此条删除。
%}}
%
%
%\begin{center}
%\textbf{ 一、第四十五条:}
%\end{center}
%
%
%\fbox{\parbox[t][5cm]{0.4\textwidth}
%{公司章程经法定代表人签字盖章生效。
%}}
%\hfill
%\fbox{\parbox[t][5cm]{0.4\textwidth}
%{公司章程经全体股东签字盖章生效。
%}}
%
%
%
%
%\textbf{二、第二十六条:}
%
%\fbox{\parbox[t][15cm]{0.45\textwidth}
%{  股东会分定期会议和临时会议。股东会每半年定期召开,由董事长召集主持。董事长不能履行或者不履行召集股东会会议职责的,由副董事长召集和主持;副董事长不能或者不履行召集和主持股东会议职责的,由监事会召集和主持;监事会不召集和主持的,代表十分之一以上表决权的股东可以自行召集和主持。}}
%\hfill
%\fbox{\parbox[t][15cm]{0.45\textwidth}
%{
%副董事长不能或者不履行主持股东会议职责的,由半数以上董事共同推举一名董事主持。
%董事会不能履行召集股东会会议职责的,由监事会(不设监事会的由监事)召集和主持;监事会或者监事不召集和主持的,代表十分之一以上表决权的股东可以自行召集和主持。召开股东会会议,应于会议召开十五日前通知全体股东(注:具体通知时间可由公司章程自定)。
%}}
%
%
%
%
%
%\textbf{四、新增第四十七条:}
%
%\begin{center}
%\begin{minipage}{0.8\textwidth}
%\definecolor{shadecolor}{rgb}{0.92,0.92,0.92} % 文本背景色用
%\begin{shaded}
%因本章程产生的或与本章程有关的争议,选择下列第(一)种方式解决:\\
%(一)提交成都仲裁委员会仲裁;\\
%(二)依法向人民法院起诉。
%\end{shaded}
% \end{minipage}
% \end{center}


